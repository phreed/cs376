\documentclass{article}
\usepackage[utf8]{inputenc}

\title{CS 376 Hybrid Systems}
\author{Fred Eisele}
\date{November 2014}

\usepackage{graphicx}
\usepackage{tikz}
\usetikzlibrary{shapes,arrows}
\usepackage{amsmath}
\usepackage{amsfonts}
\usepackage{xfrac}

\begin{document}

\maketitle

\tikzstyle{block} = [draw, fill=blue!20, rectangle, 
    minimum height=3em, minimum width=6em]


\section{Hybrid Timed Automaton}
Construct a timed automaton that produces $tick$
events in a periodic pattern.

\begin{equation}
1, 2, 3, 5, 6, 7 ,8, 10, 11, \ldots
\end{equation}

\ldots or the times between events \ldots

\begin{equation}
1, 1, 1, 2, 1, 1, 1, 2, 1, 1, \ldots
\end{equation}

That is three $tick$ events one second apart
and one $tick$ event two seconds later repeated.

\begin{figure}[h!]
%% http://cremeronline.com/LaTeX/minimaltikz.pdf
\begin{tikzpicture}[auto, node distance=4cm,>=latex']
    % We start by placing the blocks
    \node [block] (actorB)  {Actor B};
    \node [block, below left of=actorB] (actorA) {Actor A};
    \node [block, below right of=actorB] (actorC) {Actor C};

    % Once the nodes are placed, connecting them is easy. 
    \draw [->] (actorA) -- node {$1 \qquad 3$} (actorB);
    \draw [->] (actorB) -- node [name=y] {$2 \qquad 3$}(actorC);
    \draw [->] (actorA) -- node [name=y] {$1 \qquad n$}(actorC);
    
\end{tikzpicture}

\caption{Hybrid timed automata}
\label{fig:time_automata}
\end{figure}

\section{Automobile Features}

\subsection{Dome Light}
The dome light is turned on as soon as any door
is opened.
It stays on for 30 seconds after all doors are shut.
A sensor which can detect when a door's position, 
$\{opened, closed\}$ is needed for each door.

\begin{figure}[h!]

\begin{tikzpicture}[auto, node distance=4cm,>=latex']
    % We start by placing the blocks
    \node [block] (actorB)  {Actor B};
    \node [block, below left of=actorB] (actorA) {Actor A};
    \node [block, below right of=actorB] (actorC) {Actor C};

    % Once the nodes are placed, connecting them is easy. 
    \draw [->] (actorA) -- node {$1 \qquad 3$} (actorB);
    \draw [->] (actorB) -- node [name=y] {$2 \qquad 3$}(actorC);
    \draw [->] (actorA) -- node [name=y] {$1 \qquad n$}(actorC);
    
\end{tikzpicture}

\caption{A dome light}
\label{fig:dome_light}
\end{figure}


\subsection{Safety Belt Alarm}
Once the engine is runing, 
a beeper is sounded and
a red light warning is indicated if thre are 
passengers that have not buckled their seat belt.
The beeper stops sounding after 30 seconds, or
as soon as the seat belts are buckled,
whichever is sooner.
The warning light remains on so long as the
seatbelt on an occupied seat is not buckled.

The problem is made more tractable by 
replacing the engine running condition with
the ignition on condition (for gasoline engines
this is a reasonable assumption).


Each seat has two sensors, one indicating that
the $seat \in \{occupied empty\}$
and one indicating that the 
$seatbelt \in \{buckled unbuckled\}$.
Another sensor indicates 
$ignition \in \{on off\}$

Actuators $light_{warning} \in \{on off\}$
and $beeper_{warning} \in \{on off\}$ are present.



\begin{figure}[h!]

\begin{tikzpicture}[auto, node distance=4cm,>=latex']
    % We start by placing the blocks
    \node [block] (actorB)  {Actor B};
    \node [block, below left of=actorB] (actorA) {Actor A};
    \node [block, below right of=actorB] (actorC) {Actor C};

    % Once the nodes are placed, connecting them is easy. 
    \draw [->] (actorA) -- node {$1 \qquad 3$} (actorB);
    \draw [->] (actorB) -- node [name=y] {$2 \qquad 3$}(actorC);
    \draw [->] (actorA) -- node [name=y] {$1 \qquad n$}(actorC);
    
\end{tikzpicture}
\caption{Seatbelt alarm}
\label{fig:seatbelt_alarm}
\end{figure}

\end{document}
