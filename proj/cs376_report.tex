\documentclass{article}
\usepackage[utf8]{inputenc}
\usepackage{natbib}
\usepackage{amsmath}

\title{Review of A Hybrid System Model of Seasonal Snowpack Water Balance}
\author{Fred Eisele }
\date{2014 December 1}

\usepackage[showframe=true]{geometry}
\geometry{verbose, tmargin=0pt, bmargin=90pt, lmargin=90pt, rmargin=90pt}

\begin{document}

\maketitle

\section{Project Description}

Systems with long time steps may be modeled with hybrid automata
\citep{kerkez2010swb}.
This project replicates part of the work involved with modeling
the seasonal snowpack water balance and the snow-water-equivalence (SWE).
The SWE is important in estimations of the water available
for consumption in the winter snowpack.
This is of particular importance in the desert west of the USA.

\section{Progress}

I have a hybrid model based on the recommended
Stateflow\textregistered Simulink\textregistered
design patterns \citep{matlab2009ssdp}.
I am using discrete time steps (rather than continuous) as the
physical system has a large degree of variability.

\section{Problems}

Starting with a continuous model reveiled semantic
issues with embedding a continuous integrator in a SimuLink function.
Switching to a discrete integrator resolved these issues but
there are still questions related to how to appropriately
model the differential equations.

The other significant issue is the proper calibration of the model.
Typically the models produced with
Stateflow\textregistered Simulink\textregistered are working
with time frames in the one second region.
This problem uses time frames in the hour or day region.

A minor issue is that the paper does not describe
the input in much detail.
The data they use is that recorded in various surveys.
This will be approximated in my model with a sine wave
that immitates the diurnal insolation with the trough
giving up energy.

\begin{align}
dI(t) &= A ( 1.5 - \cos{ t 2 \pi } )  \\
I_{daily} &= A
\end{align}

Where $t$ is measured in days.
$A$ is the average daily insolation.
A reasonable, Springtime, value for $A$ is near $3 \text{kW-hr}/m^2$.



\bibliographystyle{plain}
\bibliography{references}

\end{document}
