\documentclass{article}
\usepackage[utf8]{inputenc}

\title{CS376 : HW 5}
\author{Fred Eisele }
\date{22 October 2014}

\usepackage{natbib}
\usepackage{graphicx}
\usepackage{tikz}
\usetikzlibrary{shapes,arrows}
\usepackage{mathtools}

\begin{document}

\renewcommand{\thesection}{\arabic{section}}
\renewcommand{\thesubsection}{\alph{subsection}}

\maketitle

\tikzstyle{block} = [draw, fill=blue!20, rectangle,
    minimum height=3em, minimum width=6em]

\setcounter{section}{5}

\section{Synchronous Data Flow}

% The block diagram code is probably more verbose than necessary
\begin{tikzpicture}[auto, node distance=4cm,>=latex']
    % We start by placing the blocks
    \node [block] (actorA) {Actor A};
    \node [block, right of=actorA] (actorB) {Actor B};
    \node [block, right of=actorB] (actorC) {Actor C};

    % Once the nodes are placed, connecting them is easy.
    \draw [->] (actorA) -- node {$1  \qquad   3$} (actorB);
    \draw [->] (actorB) -- node [name=y] {$2  \qquad    3$}(actorC);
\end{tikzpicture}

The numbers adjacent to the ports indicate the number of tokens
produced or consumed by the adjacent actor when it fires.

\subsection{Balance equations and solution}

The balance equations are \ldots

\begin{align}
\frac{q_A}{1} & = \frac{q_B}{3} \\
\frac{q_B}{2} & = \frac{q_B}{3}
\end{align}

A sufficient condition for a least positive integer solution
is to find the least common divisor which is
is to have one of the $q$ values be $1$ in this case there
is no such set of values so $2$ by mathematical induction.
The least positive integer solution set of q values
which conforms to the balance equations is:

\begin{align}
q_A & = 9  \\
q_B & = 3 \\
q_C & = 2
\end{align}

\subsection{A schedule}

The rows i are flows and the colunms are actors.
\begin{equation}
\Gamma =
\begin{bmatrix}
1 & -3 & 0 \\
0 & 2 & -3
\end{bmatrix}
\end{equation}

The schedule is
\begin{equation}
A \prec A \prec A \prec B \prec A \prec A \prec A \prec B \prec C \prec A \prec A \prec A \prec B \prec A \prec A \prec A \prec B \prec C
\end{equation}

The minimum buffer sizes are:
\begin{align}
b_{AB} & = 3  \\
b_{BC} & = 4
\end{align}


\section{Determine unbounded execution with bounded buffers}

\subsection{first}

\begin{tikzpicture}[auto, node distance=4cm,>=latex']
    % We start by placing the blocks
    \node [block] (actorB)  {Actor B};
    \node [block, below left of=actorB] (actorA) {Actor A};
    \node [block, below right of=actorB] (actorC) {Actor C};

    % Once the nodes are placed, connecting them is easy.
    \draw [->] (actorA) -- node {$1 \qquad 3$} (actorB);
    \draw [->] (actorB) -- node [name=y] {$2 \qquad 3$}(actorC);
    \draw [->] (actorA) -- node [name=y] {$1 \qquad 6$}(actorC);
\end{tikzpicture}

\subsection{second}


\begin{tikzpicture}[auto, node distance=4cm,>=latex']
    % We start by placing the blocks
    \node [block] (actorB)  {Actor B};
    \node [block, below left of=actorB] (actorA) {Actor A};
    \node [block, below right of=actorB] (actorC) {Actor C};

    % Once the nodes are placed, connecting them is easy.
    \draw [->] (actorA) -- node {$1 \qquad 3$} (actorB);
    \draw [->] (actorB) -- node [name=y] {$2 \qquad 3$}(actorC);
    \draw [->] (actorA) -- node [name=y] {$1 \qquad n$}(actorC);
\end{tikzpicture}

Where $n \in \mathbb{I}_+$.


\bibliographystyle{plain}
\bibliography{references}
\end{document}
